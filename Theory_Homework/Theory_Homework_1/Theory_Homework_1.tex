%% This is LaTeX template for preparing papers for Publ. Inst. Math.; version of 12.12.2013
%% Please delete everything begining with %% (DOUBLE %).

% Submission number: please insert
\documentclass[a4paper]{amsproc}
\usepackage{amssymb}
\usepackage{amsmath}
\usepackage{amscd} %% Package for commutative diagrams
%\usepackage[dvips]{graphicx} %% Package for inserting illustrations/figures

%% The following packages are useful (you may want to use them):
%\usepackage{refcheck} %% Checks whether enumerated equations are referred to or not.
                       %% Please remove unnecessary numbers.
%\usepackage{cmdtrack} %% Checks whether all author defined macros are used or not
                       %% (see the end of .log file); unused ones should be removed.
%% Both of the packages have some limitations---consult package documentations.

\theoremstyle{plain}
 \newtheorem*{thm}{Theorem}
 \newtheorem{prop}{Proposition}[section]
 \newtheorem{lem}{Lemma}[section]
 \newtheorem{cor}{Corollary}[section]
\theoremstyle{definition}
 \newtheorem{exm}{Example}[section]
 \newtheorem{dfn}{Definition}[section]
\theoremstyle{remark}
 \newtheorem{rem}{Remark}[section]
 \numberwithin{equation}{section}

%% Please, do not change the following four lines:
\renewcommand{\le}{\leqslant}\renewcommand{\leq}{\leqslant}
\renewcommand{\ge}{\geqslant}\renewcommand{\geq}{\geqslant}
\renewcommand{\setminus}{\smallsetminus}
\setlength{\textwidth}{28cc} \setlength{\textheight}{42cc}


% \let\oldequation=\equation
% \let\endoldequation=\endequation
% \renewenvironment{equation}{\begin{oldequation}}{\end{oldequation}\vspace{1mm}}

\providecommand{\norm}[1]{\lVert#1 \rVert}
\providecommand{\R}{\begin{pmatrix} R \\ 0 \end{pmatrix}}
\providecommand{\Q}{\begin{pmatrix} Q_1^{T} \\ Q_2^{T} \end{pmatrix}}
\providecommand{\SVD}{\begin{pmatrix} \Sigma \\ 0 \end{pmatrix}}
\providecommand{\SVDr}{\begin{pmatrix} \Sigma_1 & 0 \\ 0 & 0 \end{pmatrix}}
\providecommand{\V}{\begin{pmatrix} V_1^{T} \\ V_2^{T} \end{pmatrix}}
\providecommand{\U}{\begin{pmatrix} U_1  U_2 \end{pmatrix}}
\providecommand{\Vr}{\begin{pmatrix} v_1^{T} \\ \vdots \\ v_r^{T} \end{pmatrix}}
\providecommand{\Vn}{\begin{pmatrix} v_1^{T} \\ \vdots \\ v_n^{T} \end{pmatrix}}
\providecommand{\Vni}{\begin{pmatrix} v_{1i} \dots v_{ni}\end{pmatrix}}
\providecommand{\B}{\begin{pmatrix} b_1^{T} \\ \vdots \\ b_n^{T} \end{pmatrix}}

\title[MAT 228A Theory Homework 1]{MAT 228A Theory Homework 1}

% \subjclass[2010]{Primary REQUIRED; Secondary OPTIONAL}

%% Please use the newest classification -- 2010
%% available at  http://msc2010.org/MSC-2010-server.html
%% and the newest amsproc.cls -- from 2009!!
%% Please, classify to the third level,
%% e.g., 26A and 26Axx are not satisfsctory.

% \keywords{optional, but desirable}

\author[Cherkashin]{\bfseries Ivan Cherkashin}

% \address{
% Department of Mathematics \\ % \hfill (Received 00 00 2010)\\
% Our University   \\ %\hfill (Revised  00 00 2010)\\
% Town\\
% Country}
% \email{user@server}

%% OTHER AUTHOR(S):
%\author[]{}
%\address{ }
%\email{}

% \thanks{Partially supported by ... } %% optional

% \dedicatory{Communicated by }
%% We use this for communication information.
%% If you want do dedicate your paper to somebody, then please use \thanks{}

\begin{document}

%{\begin{flushleft}\baselineskip9pt\scriptsize
%PUBLICATIONS DE L'INSTITUT MATH\'EMATIQUE\newline
%Nouvelle s\'erie, tome 91(105) (2012), od--do \hfill DOI:
%\end{flushleft}}
\vspace{18mm} \setcounter{page}{1} \thispagestyle{empty}


% \begin{abstract}
% An abstract is OBLIGATORY!
% Please do not use author defined macros in the abstract
% and avoid references to anything in the paper,
% since the abstract will be detached from the article.
% \end{abstract}

\maketitle

\section*{0.1.1}

\begin{thm} \label{} % of course, label is optional
\begin{equation}\label{eq:0.1}
\left\lvert \frac{f(x_i)-f(x_{i-1})}{h}-f^{\prime}(x_i) \right\rvert \le Ch 
\end{equation}
where $C$ is independent of $h=x_i-x_{i-1}$, $f \in C^2[x_{i-1}, x_i]$ 
\end{thm}

\begin{proof}
By Lagrange's theorem, the Taylor series of $f(x)$ about $x_{i-1}$ can be expressed in the following way:

\begin{equation}\label{eq:0.2}
f(x_{i-1}) = f(x_i - h) = f(x_i)-f^{\prime}(x_i)h+\frac{f^{\prime \prime}(\theta)}{2}h^2
\end{equation}

where $\theta \in [x_{i-1}, x_i]$. From \eqref{eq:0.1} it follows that

\begin{equation}\label{eq:0.3}
\left\lvert \frac{f(x_i)-f(x_{i-1})}{h}-f^{\prime}(x_i) \right\rvert = \left\lvert \frac{f^{\prime \prime}(\theta)}{2} \right\rvert h
\end{equation}

from which the following estimate is true

\begin{equation}\label{eq:0.4}
\left\lvert \frac{f(x_i)-f(x_{i-1})}{h}-f^{\prime}(x_i) \right\rvert \le Ch
\end{equation}
 
where $ C = \underset{\theta \in [x_{i-1}, x_i]}{\sup} \left\lvert \frac{f^{\prime \prime}(\theta)}{2} \right\rvert = \underset{\theta \in [x_{i-1}, x_i]}{\max} \left\lvert \frac{f^{\prime \prime}(\theta)}{2} \right\rvert $.
The supremum is necessarily attained, since $f^{\prime \prime}$ is a continuous function on a compact $[x_{i-1}, x_i]$.

\end{proof}

\section*{0.1.2}

\begin{thm} \label{some label} % of course, label is optional
\begin{multline}
\label{eq:1.1} \frac{(f(x_{i+1})+E(x_{i+1})h^2)-(f(x_{i-1})+E(x_{i-1})h^2)}{2h} = \\ f^{\prime}(x_i)+O(h^2)+E^{\prime}(x_i)h^2+O(h^4) \end{multline} 
where $ h=x_i-x_{i-1}=x_{i+1}-x_{i}, ~ f,E \in C^3[x_{i-1}, x_{i+1}]$ 
\end{thm}

\begin{proof}
By Lagrange's theorem, the Taylor series of $F(x) \in C^3[x_{i-1}, x_{i+1}]$ about $x_{i+1}$ and $x_{i-1}$ can be expressed in the following way:

\begin{equation}\label{eq:1.2}
F(x_{i+1})=F(x_i)+F^{\prime}(x_i)h+\frac{F^{\prime \prime}(x_{i})}{2}h^2+\frac{F^{\prime \prime \prime}(\theta_1)}{6}h^3
\end{equation}

\begin{equation}\label{eq:1.3}
F(x_{i-1})=F(x_i)-F^{\prime}(x_i)h+\frac{F^{\prime \prime}(x_{i})}{2}h^2-\frac{F^{\prime \prime \prime}(\theta_2)}{6}h^3
\end{equation}

where $\theta_1, \theta_2 \in [x_{i-1}, x_i]$. Subtracting \eqref{eq:1.3} from \eqref{eq:1.2} and dividing by $2h$ results in 

\begin{equation}\label{eq:1.4}
\frac{F(x_{i+1})-F(x_{i-1})}{2h}=F^{\prime}(x_i)+\frac{F^{\prime \prime \prime}(\theta_1)+F^{\prime \prime \prime}(\theta_2)}{12}h^2
\end{equation}

Considering that $F(x) = f(x) + E(x)h^2$, \eqref{eq:1.4} becomes 

\begin{multline}\label{eq:1.5}
\frac{(f(x_{i+1})+E(x_{i+1})h^2)-(f(x_{i-1})+E(x_{i-1})h^2)}{2h}= \\ 
f^{\prime}(x_i)+E^{\prime}(x_i)h^2+\frac{f^{\prime \prime \prime}(\theta_1)+f^{\prime \prime \prime}(\theta_2)}{12}h^2+\frac{E^{\prime \prime \prime}(\theta_1)+E^{\prime \prime \prime}(\theta_2)}{12}h^4
\end{multline}

from which follows the estimate

\begin{multline}\label{eq:1.6}
\left\lvert \frac{(f(x_{i+1})+E(x_{i+1})h^2)-(f(x_{i-1})+E(x_{i-1})h^2)}{2h} - f^{\prime}(x_i)-E^{\prime}(x_i)h^2 \right\rvert \le \\ 
C_1 h^2+C_2 h^4
\end{multline}
 
where $ C_1 = \underset{\theta \in [x_{i-1}, x_{i+1}]}{\sup} \left\lvert \frac{f^{\prime \prime \prime}(\theta)}{12} \right\rvert $, $ C_2 = \underset{\theta \in [x_{i-1}, x_{i+1}]}{\sup} \left\lvert \frac{E^{\prime \prime \prime}(\theta)}{12} \right\rvert $.

The suprema are necessarily attained, since $f^{\prime \prime \prime}$, $E^{\prime \prime \prime}$ are continuous functions on a compact $[x_{i-1}, x_{i+1}]$.

But the equation \eqref{eq:1.6} is simply the definition of 

 \begin{multline}\label{eq:1.7} 
\frac{(f(x_{i+1})+E(x_{i+1})h^2)-(f(x_{i-1})+E(x_{i-1})h^2)}{2h} = \\ 
f^{\prime}(x_i)+E^{\prime}(x_i)h^2+O(h^2)+O(h^4)
\end{multline}

\end{proof}

\section*{0.2.1}

\begin{thm} \label{some label}
$(\forall K \in \mathbb{R_+}) \wedge (\forall n \in \mathbb{N})$ \begin{equation} 1+K\Delta t \le e^{K\Delta t} \implies (1+K\Delta t)^n \le e^{Kt}\end{equation} \\ where $ t = n\Delta t$.  
\end{thm}

\begin{proof}

Since $K\Delta t \ge 0$,

\begin{equation}\label{eq:2.1}  
1+K\Delta t \le 1+K\Delta t + \sum^{\infty}_{k = 2} \frac{(K\Delta t)^k}{k!} = \sum^{\infty}_{k = 0} \frac{(K\Delta t)^k}{k!}= e^{K\Delta t} \end{equation}

i.e.

\begin{equation}\label{eq:2.2}  
1+K\Delta t \le e^{K\Delta t} 
\end{equation}

and, hence,

\begin{equation}\label{eq:2.3} 
(1+K\Delta t)^n \le e^{Kn\Delta t} = e^{Kt} 
\end{equation}

Since $ t = n\Delta t$. 

\end{proof}

\section*{0.2.3}

\begin{thm}
 Heun's method 
 
 \begin{center}
 $ y^{n+1} = y^n + \frac{\tau}{2}[f(y^n) + f(y^n + \tau f(y^n))] $ 
 \end{center} 
 
 is a stable and consistent discretization of the initial value problem 
 
 \begin{center}
 $ y_t = f(y), ~ y(0) = y_0$ 
 \end{center}
 
 when $f \in C^2$
 
\end{thm}

\begin{proof}[Consistency]

The truncation error is

\begin{equation}\label{eq:3.1}
 \frac{1}{\tau}[y(t^{n+1}) - y(t^{n}) - \frac{\tau}{2}[f(y(t^{n})) + f(y(t^{n}) + \tau f(y(t^{n})))]]
\end{equation}

Remark that, since $f \in C ^2 \implies y \in C^3$,

\begin{equation}\label{eq:3.2}
 y(t^{n+1}) = y(t^{n} + \tau) = y(t^{n}) + y_t(t^{n}) \tau + \frac{y_{tt}(t^n) \tau^2}{2} + O(\tau^3)
\end{equation}

and that

\begin{equation}\label{eq:3.3}
 f(y(t^{n}) + \tau f(y(t^{n}))) = f(y(t^{n})) + f_y(y(t^{n})) f(y(t^{n})) \tau + O(\tau^2)
\end{equation}

Relations \eqref{eq:3.2} and \eqref{eq:3.3} imply, after reordering of terms, that the truncation error is

\begin{equation}\label{eq:3.4}
 [y_t(t^{n}) - f(y(t^{n}))] + \frac{\tau}{2}[y_{tt}(t^n) - f_y(y(t^{n})) f(y(t^{n})) ] + O(\tau^2)
\end{equation}

However, it is true that

\begin{equation}\label{eq:3.5}
 y_t(t^{n}) - f(y(t^{n})) = 0
\end{equation}

which is the statement of the initial value problem, and that

\begin{multline}\label{eq:3.6}
 y_{tt}(t^{n}) = f_t(y(t^{n})) = f_y(y(t^{n})) y_t(t^{n}) = f_y(y(t^{n})) f(y(t^{n})) \\ \implies y_{tt}(t^n) - f_y(y(t^{n})) f(y(t^{n})) = 0
\end{multline}

Thus, from \eqref{eq:3.5} and \eqref{eq:3.6} it follows that the truncation error is of second order:

\begin{equation}\label{eq:3.7}
 O(\tau^2)
\end{equation}

and, therefore, Heun's method is consistent.

\end{proof}

\begin{proof}[Stability]

Let
  \begin{equation}\label{eq:3.8} 
 F(y) = \frac{f(y)  + f(y + \tau f(y))}{2}
 \end{equation}
 
  Also, $f \in C^2 \implies F \in C^2$.
 
 Since $\mathbb{R}$ is locally compact, any two states $y_1 < y_2 < \infty $ belong to some compact $K \subset \mathbb{R}$. If the solution is sought only for such time intervals during which the solution does not escape the compact $K$, then $F \in C^2(K) \implies F \in Lip(K)$, which means that
 
  \begin{equation}\label{eq:3.9}
 \forall y_1, y_2 \in K, ~ 
 \lvert F(y_2)- F(y_1) \rvert \le C \lvert y_2 - y_1 \rvert
 \end{equation}
 
 where the Lipschitz constant $C = \underset{\theta \in K}{\sup} \lvert F_y(\theta) \rvert $, by Lagrange's theorem. The supremum is attained because $F_y$ is continuous on a compact. 
 
 Now, consider how the approximation operator "repels" the two states:
 
 \begin{multline}\label{eq:3.10}
  \lvert L_{\tau}[y_2] - L_{\tau}[y_1] \rvert = \lvert (y_2 - y_1) + \tau(F(y_2)- F(y_1)) \rvert \le \lvert y_2 - y_1 \rvert + \tau \lvert F(y_2)- F(y_1) \rvert 
  \\ \le (1 + \tau C) \lvert y_2 - y_1 \rvert
 \end{multline}

 But that is exactly the definition of stability. Thus, the notion of stability of an approximation operator is the same as the notion of compactness of an operator: an operator is called \emph{compact} if the image of a bounded set is a totally bounded set (i.e. the image of a bounded set is precompact). Since precompact sets are bounded, compact operators preserve boundedness of sets. But that is exactly how stability is understood: it is when a bounded region of uncertainty in input remains bounded after the approximation operator is applied. 
 
\end{proof}

\begin{proof}[Convergence]
 
 Since Heun's method is stable and consistent when $f \in C^2$, by Theorem 0.2.2 it converges to the exact solution of the initial value problem stated in the theorem. By the same theorem, the convergence of Heun's method is of \emph{second order}, which is the same order as the order of the truncation error.
 
\end{proof}



\section*{1.1.1}

\begin{thm} \label{some label}
The centered difference method $u^{n+1}_i = u^{n}_i+\frac{\sigma}{2}(u^{n}_{i-1}-u^{n}_{i+1}) $ is unstable in the discrete max norm $\norm{\cdot}_{\infty}$  
\end{thm}

\begin{proof}

The discrete max norm and the discrete $L^2$ norm are related thus:

\begin{equation}\label{eq:4.1}
 \norm{x}_{\infty} \le \norm{x}_{2} \le \sqrt{n}\norm{x}_{\infty}  
\end{equation}

where $x \in \mathbb{R}^n$. This relationship also means that the discrete max norm and the discrete $L^2$ norm induce the same topology on $\mathbb{R}^n$. In fact, in finite dimensional linear spaces, all norms induce the same topology. Thus, the stability of a finite difference scheme is a topological property. This make sense, because the spectrum of the approximation operator, which determines the stability of a finite difference scheme, is a topological invariant.

Since the centered difference method is unstable in the $L^2$ norm, from \eqref{eq:4.1} it follows that it is also unstable in the discrete max norm $\norm{\cdot}_{\infty}$.

NOTE: is the centered difference scheme \emph{always} unstable? 

Counter-example:

\begin{equation}\label{eq:4.2}
\underset{\phi \in S}{\max} \lvert \lambda(\phi) \rvert = \sqrt{1+\sigma^2} = \sqrt{1+(\frac{a\tau}{h})^2} 
\end{equation}

If $\tau = Ah^2$, $A \in R_+$ while $\tau, h \rightarrow 0$, then

\begin{equation}\label{eq:4.3} 
\underset{\phi \in S}{\max} \lvert \lambda(\phi) \rvert = \sqrt{1+(\frac{a\tau}{h})^2} = \sqrt{1+a^2A\tau} = 1 + \frac{a^2A\tau}{2} + O(\tau^2) \end{equation}

\begin{equation}\label{eq:4.4} 
\underset{\phi \in S}{\max} \lvert \lambda(\phi) \rvert = 1 + \frac{a^2A\tau}{2} + O(\tau^2) \le e^{C\tau}
\end{equation}

where $C < \infty$ is a constant. Therefore, by spectral stability criterion, the centered difference scheme must be stable if $\tau = Ah^2$, $A \in R_+$ while $\tau, h \rightarrow 0$. However, this scheme is not preferred practically, since the stability requirement on the time step $\tau = O(h^2)$ is too rigid.
                                                                                                                  
\end{proof}

% \bibliographystyle{amsplain}
% \begin{thebibliography}{n} %% n is number of items, or the largest label
% 
% \bibitem{1}\label{some label - optional} A.\,U. Thor, (not Thor, A.U.!)
% \emph{Title of paper},
% J. Math. \textbf{99} (2008), 111--222.
% 
% \bibitem{2} A.\,U. Thor,
% \emph{Title of paper},
% in: E. Ditor (ed.), \emph{Title of Book}, Publisher, City, Year, 888--999.
% 
% \end{thebibliography}

\end{document}

%% To be filled in the journal office:

@author:
@affiliation:
@title:
@language: English
@pages:
@classification1:
@classification2:
@keywords:
@abstract:
@filename:
@EOI


